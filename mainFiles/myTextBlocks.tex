%% INTRODUCTION
This document describes the usage of the MATLAB class {\tt latexGenerator}.
It is based on the super class {\tt reportGenerator}. Another availbale sub class
is {\tt markdownGenerator} which instead of Latex produces output for the
Markdown language.

This file is produced by the MATLAB File {\tt testLatexGenerator}.
Necessary files are:

\begin{description}
\item[{\tt setting.dat}] Specifies a variaty of settings for the program.

You will have to modify {\tt outputGenerator} and {\tt outputViewer} which 
in our example are specified for the operating system of the Mac.

\item[{\tt template.tex}] A template file for the Latex environment

Make sure you have Latex installed on your system and that it is accessible 
through MATLAB. It might be necessary to modify your file {\tt startup.m} 
to allow MATLAB to run Latex programs.

\item[{\tt myTextBlocks.tex}] A file which contains blocks of Latex code 
separated by lines with {\tt \%\% TAG} where {\tt TAG} is a unique identifier.

It can be used to include Latex code into the document. Please see at the 
end of this chapter how the text of this introduction is included into the document.

\item[{\tt myCodeBlocks.m}] A file which contains blocks of MATLAB code 
separated by lines with {\tt \%\% TAG} where {\tt TAG} is a unique identifier.

It can be used to evaluate code specified in a section of this file. 

\item[{\tt myMatlabInit.m}] This is a init-File for MATLAB which is executed 
at the beginning. One can specify all necessarey settings there.

\end{description}

%% LIST
Here it is shown how the three Latex environments for lists can be
specified within MATLAB code.

Of course one does not have to do this because a list environment 
(as any Latex code) can be imported from text blocks (see e.g. Introduction).

%% TEXT2
Lorem ipsum dolor sit amet, consetetur sadipscing elitr, sed diam nonumy 
eirmod tempor invidunt ut labore et dolore magna aliquyam erat, sed diam 
voluptua. At vero eos et accusam et justo duo dolores et ea rebum. 